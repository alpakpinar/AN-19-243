\section{Introduction} \label{section:introduction}

Several astrophysical observations \cite{Bertone:2004pz,Feng:2010gw,Porter:2011nv} 
provide compelling evidence for the existence of
dark matter (DM), a type of matter not accounted for in the standard model (SM).
To date only gravitational interactions of DM have been observed and it remains unknown
if DM has a particle origin and could interact with ordinary matter via SM processes. However,
many theoretical models have been proposed in which DM and SM
particles interact with sufficient strength that DM may be directly produced with
observable rates in high energy collisions at the CERN LHC.
While the DM particles would remain undetected, they may recoil
with large transverse momentum (\pt) against other detectable particles
resulting in an overall visible \pt imbalance in a collision event. This type of event topology is
rarely produced in SM processes and therefore enables a highly sensitive search 
for DM. Similar event topologies are predicted by other extensions of the SM, 
such as the Arkani-Hamed, Dimopoulos, and Dvali (ADD)
model ~\cite{bib:ADD1,ADDPRD,Antoniadis,ADDGiudice,ADDPeskin} of large extra spatial dimensions (EDs).

This analysis note describes a search for Higgs boson produced by vector boson fusion (VBF), 
decaying into invisible particles. The signature of such a final state will be two seperated jets 
and an imbalance in \vpt due to the undetected particles. Two seperated jets are the result of the 
hadronization of two final state quarks emerging from the VBF process.

This analysis makes use of data collected with the CMS detector in proton-proton (pp) collisions in 2017 
at $\sqrt{s}=13\TeV$, corresponding to an integrated luminosity of 41.5 \fbinv. 


This paper describes a search for new physics resulting in final
states with one or more energetic jets and an imbalance in \pt
due to undetected particles. The jets are the result of the fragmentation and hadronization of
quarks or gluons, which may be produced directly in the hard scattering process as initial-state radiation or
as the decay products of a vector boson $\PV$ ($\PW$ or $\PZ$).
These final states are commonly referred to as ``monojet'' and ``mono-$\PV$''.
Several searches have been performed at the LHC using the
monojet and mono-$\PV$ channels~\cite{Aad:2013oja,Khachatryan:2014rra,Aad:2015zva,Khachatryan:2016mdm,Aaboud:2016tnv,paper-exo-037,Aaboud:2017phn}.
This analysis makes use of a data sample of proton-proton
(pp) collisions at $\sqrt{s}=13\TeV$ collected with the
CMS detector at the LHC, corresponding to an integrated
luminosity of 101.3 \fbinv. This sample is approximately
three times larger than the one used in Ref.~\cite{Sirunyan:2017jix}.
The analysis strategy is similar to that of previous CMS searches, and simultaneously
employs event categories to target both the monojet and mono-$\PV$ final states.
In an improvement compared to previous searches, in this paper revised theoretical predictions and uncertainties
for \phojets, \Zjets, and \Wjets processes based on recommendations of Ref.~\cite{DMTheory} are used.
