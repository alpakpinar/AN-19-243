\section{Samples} \label{sec:samples}

The analysis described in this note is performed using data collected in 2017 by
CMS at 13 TeV and corresponds to an integrated luminosity of 41.5 fb$^{-1}$.
The MC simulation samples for the background processes
have been produced in the \texttt{Fall17} and \texttt{Autumn18} campaigns. Further details
are given in the following sub-sections.

\subsection{Data Samples}

The datasets listed in Tab.~\ref{tab:DataSamples} are used to
select events in the signal and the control regions, and to perform measurments on physics objects
used in the analysis (e.g. trigger turn-ons).

\begin{table}[ht!]
    \centering
    \small
    \def\arraystretch{1.5}
    \caption{List of datasets used to select events in the signal and control regions. Datasets for both years correspond to the \texttt{Nano1June2019} campaign, otherwise known as \texttt{v5}. For the 2017 data, the \texttt{31Mar2018} reconstruction is used for all periods. For 2018 data, the \texttt{17Sep2018} reconstruction is used for runs A to C, and the \texttt{22Jan2019} reconstruction is used for run D.}
    \begin{tabular}{l l p{8cm}}
        \hline
        \hline
        Year                  & Dataset name                       & Events selected for                                                     \\
        \hline
        \hline
        both                  & {/MET/Run201*/NANOAOD}             & Signal , single muon, double muon control regions                 \\\hline
        \multirow{2}{*}{2017} & {/SingleElectron/Run2017*/NANOAOD} & Single electron, double electron control regions                        \\
                              & {/SinglePhoton/Run2017*/NANOAOD}   & Single photon control region                                            \\\hline
        2018                  & {/EGamma/Run2018*/NANOAOD}         & Single electron, double electron control, single photon control regions \\
        \hline
        \hline
    \end{tabular}

    \label{tab:DataSamples}
\end{table}

Events for the signal region are collected using a set of dedicated
triggers designed to select events with large \ptmiss and large \mht based on
the online particle flow (PF) algorithm. In these dedicated trigger algorithms,
identified PF muons are removed from the event before the
\ptmiss~and the \mht~objects are calculated. With this definition,
the signal trigger paths can also be used to select single and double muon events for the W and Z control regions, respectively.

Electron events for the W and Z regions are selected using a single electron trigger.
To ensure the trigger efficiency also for high-\pt electrons, the single electron trigger is used in combination with
a single photon trigger~\cite{CMS-EGM-TWIKI-HLT}. The same photon trigger is used to select events for the photon control region.

The full list of triggers used, along with the L1 seeds and the associated primary datasets are shown in Table~\ref{tab:triggers}.

\begin{table}[h]
    \centering
    \def\arraystretch{1.5}

    \small
    \caption{HLT paths and the associated L1 seeds used in the analysis for the 2017 and 2018 datasets. The HLT paths ending in ``\_HT60'' are backup triggers introduced  to mitigate noise rate problems in 2017. Their inclusion is not strictly necessary for 2018, but is done for consistency.}

    \footnotesize
    \begin{tabular}{l l c c}
        \hline\hline
        Year                   & HLT path                                                  & L1 seed                         & Primary dataset               \\\hline\hline
        \multirow{5}{*}{2017}  & HLT\_PFMETNoMu120\_PFMHTNoMu120\_IDTight                  & \texttt{L1\_ETMHF70}            & MET                           \\
                               & HLT\_PFMETNoMu120\_PFMHTNoMu120\_IDTight\_PFHT60          & \texttt{L1\_ETMHF80\_HTT60er }  & MET                           \\\cline{2-4}
                               & HLT\_Ele35\_WPTight\_Gsf                                  & \texttt{L1\_SingleEG24}         & SingleElectron                \\\cline{2-4}
                               & \multirow{3}{*}{HLT\_Photon200}                           & \texttt{L1\_SingleEG30}         & \multirow{3}{*}{SinglePhoton} \\
                               &                                                           & \texttt{L1\_SingleJet170}       &                               \\
                               &                                                           & \texttt{L1\_SingleTau100er2p1}  &                               \\\hline\hline

        \multirow{11}{*}{2018} & \multirow{2}{*}{HLT\_PFMETNoMu120\_PFMHTNoMu120\_IDTight} & \texttt{L1\_ETMHF100}           & \multirow{3}{*}{MET}          \\
                               &                                                           & \texttt{L1\_ETM150}             &                               \\
                               & HLT\_PFMETNoMu120\_PFMHTNoMu120\_IDTight\_PFHT60          & \texttt{L1\_ETMHF90\_HTT60er}   &                               \\\cline{2-4}
                               & \multirow{3}{*}{HLT\_Ele32\_WPTight\_Gsf}                 & \texttt{L1\_SingleIsoEG24er2p1} & \multirow{3}{*}{EGamma}       \\
                               &                                                           & \texttt{L1\_SingleEG26er2p5}    &                               \\
                               &                                                           & \texttt{L1\_SingleEG60}         &                               \\\cline{2-4}

                               & \multirow{5}{*}{HLT\_Photon200}                           & \texttt{L1\_SingleEG34er2p5}    & \multirow{5}{*}{EGamma}       \\
                               &                                                           & \texttt{L1\_SingleJet160er2p5}  &                               \\
                               &                                                           & \texttt{L1\_SingleJet180}       &                               \\
                               &                                                           & \texttt{L1\_SingleTau120er2p1}  &                               \\
                               &                                                           & \texttt{L1\_SingleEG60}         &                               \\\hline

        \hline\hline %--------------------------------------------------------------------------------------------------------------------------      \
    \end{tabular}

    \label{tab:triggers}
\end{table}

\subsection{Background Samples}


Simulation datasets for the background processes are listed in
Table~\ref{tab:BackgroundSamples} and~\ref{tab:BackgroundSamples_2}.
There are several Standard Model processes that pose as backgrounds to the VBF $H_{inv}$ signal, experimental signature of the
final state being two jets with large rapidity seperation and invariant mass, along with \ptmiss.
These processes are as follows:

\begin{description}
\item[\Zvvjets]: This process yields the largest irreducible background in the analysis, consisting of a Z boson and 2 or more
jets coming from either QCD or EWK vertices. Simulated samples for this background have been produced at leading
order (LO) in QCD using the aMC@NLO generator in several bins of \Ht for the QCD case, and in one inclusive sample for the EWK case.

\item[\Wjets]: This process is the second largest source of background in this analysis, consisting of a W boson and 2 or more
jets coming from either QCD or EWK vertices. The contribution of this background can be reduced by rejecting events with
charged lepton candidates (electron/muon/tau).
However, this process becomes irreducible in the case where the charged leptons are outside of the detector acceptance.
Simulation samples for this background have been generated at LO in QCD using the aMC@NLO generator in several bins \Ht for the QCD case, and in once inclusive sample for the EWK case.

\item[\Zlljets]: This process mimics signal-like events in the case where the leptons coming from the Z boson decay are
not reconstructed. As in the case of \Wlv, the contribution of this background is reduced by rejecting events with charged leptons.
Simulation samples for this process have been generated at LO in QCD using the aMC@NLO generator in bins of \Ht for the QCD case, and in once inclusive sample for the EWK case.

\item[Top:] Top-quark decays (both \ttbar and single top) also contribute background events to this analysis.
In these processes, the W boson produced in a top-quark decay further decays leptonically, which produces genuine \ptmiss in the event.
Next-to-leading order (NLO) \ttbar simulation samples have been produced with the aMC@NLO generator with two additional partons in the matrix element.
Single-top events have been generated with the Powheg generator at NLO in QCD with one additional matrix element parton.

\item[Dibosons:] Decays of diboson pairs (WW, WZ, ZZ) also constitute background processes.
Typically, one of the bosons decays leptonically (\Wlv,\Zvv) while the other boson decays hadronically, thus producing jets and \ptmiss~in the final state.
Simulated samples for WW, WZ and ZZ production have been generated at LO with Pythia~8.

\item[QCD Multijet:] QCD multijet events typically do not have large genuine \ptmiss.
However, given the large cross section with which these events are produced, even a small fraction of events with
jet mismeasurement results in a non-zero contribution of this process as background in the analysis.
Simulated QCD samples have been generated at LO in QCD using the MadGraph generator in several bins of \Ht.

\end{description}

\begin{table}[ht!]
\centering
\scriptsize
    \def\arraystretch{1.3}
\begin{tabular}{l|r|c}
\hline
\hline
Dataset name                                                                  &  Cross section (pb)          & Order in QCD \\
\hline
\hline
WJetsToLNu\_HT-70To100\_TuneCP5\_13TeV-madgraphMLM-pythia8                        &   1296         & LO  \\
WJetsToLNu\_HT-100To200\_TuneCP5\_13TeV-madgraphMLM-pythia8                       &   1392         & LO  \\
WJetsToLNu\_HT-200To400\_TuneCP5\_13TeV-madgraphMLM-pythia8                       &    410.2       & LO  \\
WJetsToLNu\_HT-400To600\_TuneCP5\_13TeV-madgraphMLM-pythia8                       &     57.95      & LO  \\
WJetsToLNu\_HT-600To800\_TuneCP5\_13TeV-madgraphMLM-pythia8                       &     12.98      & LO  \\
WJetsToLNu\_HT-800To1200\_TuneCP5\_13TeV-madgraphMLM-pythia8                      &      5.39      & LO  \\
WJetsToLNu\_HT-1200To2500\_TuneCP5\_13TeV-madgraphMLM-pythia8                     &      1.08      & LO  \\
WJetsToLNu\_HT-2500ToInf\_TuneCP5\_13TeV-madgraphMLM-pythia8                      &      0.008098  & LO  \\
\hline
ZJetsToNuNu\_HT-100To200\_13TeV-madgraph                                         &    305.3       & LO  \\
ZJetsToNuNu\_HT-200To400\_13TeV-madgraph                                         &     91.86      & LO  \\
ZJetsToNuNu\_HT-400To600\_13TeV-madgraph                                         &     13.13      & LO  \\
ZJetsToNuNu\_HT-600To800\_13TeV-madgraph                                         &      3.242     & LO  \\
ZJetsToNuNu\_HT-800To1200\_13TeV-madgraph                                        &      1.501     & LO  \\
ZJetsToNuNu\_HT-1200To2500\_13TeV-madgraph                                       &      0.3431    & LO  \\
ZJetsToNuNu\_HT-2500ToInf\_13TeV-madgraph                                        &      0.005146  & LO  \\
\hline
DYJetsToLL\_M-50\_HT-70to100\_TuneCP5\_13TeV-madgraphMLM-pythia8                   &    146.9       & LO  \\
DYJetsToLL\_M-50\_HT-100to200\_TuneCP5\_13TeV-madgraphMLM-pythia8                  &    160.9       & LO  \\
DYJetsToLL\_M-50\_HT-200to400\_TuneCP5\_13TeV-madgraphMLM-pythia8                  &     48.68      & LO  \\
DYJetsToLL\_M-50\_HT-400to600\_TuneCP5\_13TeV-madgraphMLM-pythia8                  &      6.998     & LO  \\
DYJetsToLL\_M-50\_HT-600to800\_TuneCP5\_13TeV-madgraphMLM-pythia8                  &      1.745     & LO  \\
DYJetsToLL\_M-50\_HT-800to1200\_TuneCP5\_13TeV-madgraphMLM-pythia8                 &      0.8077    & LO  \\
DYJetsToLL\_M-50\_HT-1200to2500\_TuneCP5\_13TeV-madgraphMLM-pythia8                &      0.1923    & LO  \\
DYJetsToLL\_M-50\_HT-2500toInf\_TuneCP5\_13TeV-madgraphMLM-pythia8                 &      0.003477  & LO  \\
\hline
GJets\_HT-40To100\_TuneCP5\_13TeV-madgraphMLM-pythia8                             &  18640         & LO  \\
GJets\_HT-100To200\_TuneCP5\_13TeV-madgraphMLM-pythia8                            &   8641         & LO  \\
GJets\_HT-200To400\_TuneCP5\_13TeV-madgraphMLM-pythia8                            &   2196         & LO  \\
GJets\_HT-400To600\_TuneCP5\_13TeV-madgraphMLM-pythia8                            &    258.4       & LO  \\
GJets\_HT-600ToInf\_TuneCP5\_13TeV-madgraphMLM-pythia8                            &     85.23      & LO  \\
\hline
\hline
\end{tabular}
\caption{Simulated datasets for the modelling of single electroweak boson backgrounds.
All datasets are accessed at the NanoAOD data tier from the \texttt{v5} campaign, also known as \texttt{1Jun19}.}
\label{tab:BackgroundSamples}
\end{table}



\begin{table}[ht!]
    \centering
    \scriptsize
        \def\arraystretch{1.3}
    \begin{tabular}{l|r|c}
    \hline
    \hline
    Data set name                                                                  &  Cross section (pb)          & Order in QCD \\
    \hline
    \hline

    EWKWMinus2Jets\_WToLNu\_M-50\_TuneCP5\_13TeV-madgraph-pythia8                      &     20.35      & LO  \\
    EWKWPlus2Jets\_WToLNu\_M-50\_TuneCP5\_13TeV-madgraph-pythia8                       &     25.81      & LO  \\
    EWKZ2Jets\_ZToLL\_M-50\_TuneCP5\_13TeV-madgraph-pythia8                            &      4.321     & LO  \\
    EWKZ2Jets\_ZToNuNu\_TuneCP5\_13TeV-madgraph-pythia8                               &     10.04      & LO  \\
    AJJ\_EWK\_TuneCP5\_13TeV\_amcatnlo-pythia8                                         &      6.096     & NLO \\
    GJets\_Mjj-500\_SM\_5f\_TuneCP5\_EWK\_13TeV-madgraph-pythia8                         &      4.937     & LO  \\
    GJets\_SM\_5f\_TuneCP5\_EWK\_13TeV-madgraph-pythia8                                 &     32.91      & LO  \\

    \hline
    TTJets\_TuneCP5\_13TeV-amcatnloFXFX-pythia8                                      &    831.76      & NLO \\
    \hline
    ST\_t-channel\_top\_4f\_inclusiveDecays\_TuneCP5\_13TeV-powhegV2-madspin-pythia8     &    137.458     & NLO \\

    ST\_t-channel\_antitop\_4f\_inclusiveDecays\_TuneCP5\_13TeV-powhegV2-madspin-pythia8 &     83.0066    & NLO \\
    ST\_tW\_top\_5f\_inclusiveDecays\_TuneCP5\_*\_13TeV-powheg-pythia8                      &     35.85      & NLO \\
    ST\_tW\_antitop\_5f\_inclusiveDecays\_TuneCP5\_*\_13TeV-powheg-pythia8        &     35.85      & NLO \\
    \hline
    WW\_TuneCP5\_13TeV-pythia8                                                       &     75.91      & LO  \\
    WZ\_TuneCP5\_13TeV-pythia8                                                       &     27.56      & LO  \\
    ZZ\_TuneCP5\_13TeV-pythia8                                                       &     12.14      & LO  \\
    \hline
    QCD\_HT1000to1500\_TuneCP5\_13TeV-madgraph-pythia8                                &   1095         & LO  \\
    QCD\_HT1000to1500\_TuneCP5\_13TeV-madgraph-pythia8                                &   1095         & LO  \\
    QCD\_HT100to200\_TuneCP5\_13TeV-madgraph-pythia8                                  &      2.369e+07 & LO  \\
    QCD\_HT1500to2000\_TuneCP5\_13TeV-madgraph-pythia8                                &     99.27      & LO  \\
    QCD\_HT2000toInf\_TuneCP5\_13TeV-madgraph-pythia8                                 &     20.25      & LO  \\
    QCD\_HT200to300\_TuneCP5\_13TeV-madgraph-pythia8                                  &      1.554e+06 & LO  \\
    QCD\_HT300to500\_TuneCP5\_13TeV-madgraph-pythia8                                  & 324300         & LO  \\
    QCD\_HT300to500\_TuneCP5\_13TeV-madgraph-pythia8                                  & 324300         & LO  \\
    QCD\_HT500to700\_TuneCP5\_13TeV-madgraph-pythia8                                  &  29990         & LO  \\
    QCD\_HT50to100\_TuneCP5\_13TeV-madgraphMLM-pythia8                                &      1.85e+08  & LO  \\
    QCD\_HT700to1000\_TuneCP5\_13TeV-madgraph-pythia8                                 &   6374         & LO  \\
    QCD\_HT700to1000\_TuneCP5\_13TeV-madgraph-pythia8                                 &   6374         & LO  \\

    \hline
    \hline
    \end{tabular}
    \caption{Simulated datasets for the modelling of EWK V production and other processes. All datasets are accessed at the NanoAOD data tier from the \texttt{v5} campaign, also known as \texttt{1Jun19}.}
    \label{tab:BackgroundSamples_2}
\end{table}
The MC samples produced using \MGvATNLO, and \POWHEG
generators are interfaced with \PYTHIA using the CP5 tune~\cite{Sirunyan:2019dfx}
for the fragmentation, hadronization, and underlying event description.
In the case of the \MGvATNLO samples, jets from the matrix element calculations
are matched to the parton shower description following the MLM~\cite{Mangano:2006rw} (FxFx~\cite{Frederix:2012ps})
prescription to match jets from matrix element calculations and parton shower description for LO (NLO) samples.
The NNPDF 3.1 NNLO~\cite{Ball:2017nwa}
parton distribution functions (PDFs) are used in all generated samples.
The propagation of all final-state particles through the CMS detector
is simulated with the \GEANT4 software~\cite{Agostinelli:2002hh}.
The simulated events include the effects of pileup, with the multiplicity of reconstructed primary
vertices matching that in data. The average number of pileup interactions per proton bunch crossing
is found to be 32 for both the 2017 and 2018 data samples used in this analysis (assuming a total inelastic proton-proton cross-section).

\subsection{Signal simulation samples}

Simulated signal samples for invisible decays of the Higgs boson are generated using the Powheg generator at NLO in QCD.
The samples are normalized to the standard model cross section for Higgs production in the given mode~\cite{deFlorian:2016spz}.
They are listed in Tab.~\ref{tab:SignalSamples}.

\begin{table}[ht!]
    \centering
    \scriptsize
        \def\arraystretch{1.3}
    \begin{tabular}{l|r|c}
    \hline
    \hline
    Data set name                                                                  &  Cross section (pb)          & Order in QCD \\
    \hline
    \hline
GluGlu\_HToInvisible\_M125\_TuneCP5\_13TeV\_powheg\_pythia8                          &     48.58      & NLO \\
VBF\_HToInvisible\_M125\_13TeV\_TuneCP5\_powheg\_pythia8                             &      3.782     & NLO \\

\hline
\hline
\end{tabular}
\caption{Simulated signal datasets for the modelling of invisible Higgs boson decays. All datasets are accessed at the NanoAOD data tier from the \texttt{v5} campaign, also known as \texttt{1Jun19}.}
\label{tab:SignalSamples}
\end{table}