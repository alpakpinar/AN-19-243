\section{Reweighting of simulated events} \label{sec:reweighting}

Simulated signal and background samples are corrected for various effects through reweighting procedures outlined in this section.

\subsection{Trigger efficiency reweighting}

The efficiency is calculated as a function of the recoil and $\mjj$.
The trigger is found to be more than $95\%$ efficient for events with a recoil larger than $250~\GeV$, and more than $99\%$ 
efficient for events with a recoil larger than $375~\GeV$. The MC-to-data scale factor is found to be within $1\%$ of unity 
everywhere except for the lowest recoil bin at $250~\GeV$, where it is within $2\%$ (c.f. sec. ~\ref{sec:efficiency}). 

\subsection{Pileup reweighting}

The pileup (PU) conditions in the simulated samples are not identical to the ones observed measured in data, and a reweighting is applied to remove the difference.
The reweighting is performed by matching the true pileup distribution of each simulated sample
with the pileup distribution in data, obtained through the pileupCalc tool
assuming a minimum bias cross section of 69.2$\pm 4.6\%$~mb, following the recommendations in in Ref.~\cite{pileup_twiki}. The true pileup distributions in data and simulation are shown in Fig.~\ref{fig:purwg_true}. The distribution of the number of reconstructed vertices for $W\to \mu\nu$ events before and after PU reweighting is shown in Fig.~\ref{fig:purwt_npv}. In this variable, the PU reweighting method leads to a worse overall agreement between data and simulation. To check this behavior, the distribution of the event energy density $\rho$ is shown in Fig.~\ref{fig:purwt_npv}, again before and after PU reweighting. Here, the agreement before PU reweighting is worse than in the primary vertex distribution and the PU reweighting clearly improves the agreement.

\begin{figure}[ht!]
  \begin{center}
    \includegraphics[width=0.49\textwidth]{fig/pileup/pu_weights_2017.pdf}
    \includegraphics[width=0.49\textwidth]{fig/pileup/pu_weights_2018.pdf}
    \caption{
        Distribution of the true number of PU events in data and simulation for 2017 (left) and 2018 (right).
        The distributions for data are extracted assuming a minimum bias cross section of $69.2~\mathrm{mb}$.
    }
    \label{fig:purwg_true}
  \end{center}
\end{figure}
\begin{figure}[ht!]
  \begin{center}
    \includegraphics[width=0.49\textwidth]{fig/pileup/cr_1m_j_npv_nopu_2017.pdf}
    \includegraphics[width=0.49\textwidth]{fig/pileup/cr_1m_j_npv_2017.pdf}
    \includegraphics[width=0.49\textwidth]{fig/pileup/cr_1m_j_npv_nopu_2018.pdf}
    \includegraphics[width=0.49\textwidth]{fig/pileup/cr_1m_j_npv_2018.pdf}
    \caption{
        Distribution of the number of vertices in \Wmn~events in data and
        simulation before pileup re-weighting (left) and after pileup reweighting (right).
        The Monte Carlo is normalized to the luminosity of 41.53 and 59.7 fb$^{-1}$, respectively for 2017 and 2018.
    }
    \label{fig:purwt_npv}
  \end{center}
\end{figure}
\begin{figure}[ht!]
  \begin{center}
    \includegraphics[width=0.49\textwidth]{fig/pileup/cr_1m_j_rho_all_nopu_2017.pdf}
    \includegraphics[width=0.49\textwidth]{fig/pileup/cr_1m_j_rho_all_2017.pdf}
    \includegraphics[width=0.49\textwidth]{fig/pileup/cr_1m_j_rho_all_nopu_2018.pdf}
    \includegraphics[width=0.49\textwidth]{fig/pileup/cr_1m_j_rho_all_2018.pdf}
    \caption{
        Distribution of the event energy density $\rho$  in \Wmn~events in data and
        simulation before pileup re-weighting (left) and after pileup reweighting (right).
        The Monte Carlo is normalized to the luminosity of 41.53 and 59.7 fb$^{-1}$, respectively for 2017 and 2018.
    }
    \label{fig:purwt_rho}
  \end{center}
\end{figure}

\subsection{Lepton and photon identification/reconstruction efficiency reweighting}

Data-to-simulation scale factors are applied to events in the control regions to
account for differences in the reconstruction, identification and isolation of leptons
between data and simulationn. These data-to-MC scale factors are derived from the efficiencies that are measured for the electron and muon
selections in bins of $\pt$ and $\eta$ in both data and simulation. These scale factors are
provided by the relevant POGs.


The reconstruction scale factors for electrons are shown in Fig.~\ref{fig:sf_electron_reco}. The corresponding identification scale factors for veto and tight electrons are shown in Fig.~\ref{fig:sf_electron_id}, and include the effect of the isolation efficiency. 

The identification scale factors for muons are shown in Fig.~\ref{fig:sf_muon_id}. Here, isolation scale factors are applied separately and are shown in Fig.~\ref{sf_muon_iso}. The corresponding corrections for muons are deemed negligible~\cite{CMS-MUO-TWIKI-SF}.

\begin{figure}[ht!]
  \begin{center}
    \includegraphics[width=0.49\textwidth]{fig/efficiency/lepton/ele_eff_reco_2017.pdf}
    \includegraphics[width=0.49\textwidth]{fig/efficiency/lepton/ele_eff_reco_2017.pdf}\\
    \caption{
      Scale factors for the reconstruction efficiency of electrons starting from a super cluster for 2017 (left) and 2018(right)
    }
    \label{fig:sf_electron_reco}
  \end{center}
\end{figure}
\begin{figure}[ht!]
  \begin{center}
    \includegraphics[width=0.49\textwidth]{fig/efficiency/lepton/ele_eff_tight_id_2017.pdf}
    \includegraphics[width=0.49\textwidth]{fig/efficiency/lepton/ele_eff_loose_id_2017.pdf}\\
    \includegraphics[width=0.49\textwidth]{fig/efficiency/lepton/ele_eff_tight_id_2018.pdf}
    \includegraphics[width=0.49\textwidth]{fig/efficiency/lepton/ele_eff_loose_id_2018.pdf}
    \caption{
      Scale factors for tight (left) and veto (right) electrons are shown for 2017 (top) and
      2018 (bottom). The scale factors are provided in bins of electron $\pt$ and $\eta$.
    }
    \label{fig:sf_electron_id}
  \end{center}
\end{figure}

The scale factors for id and isolation for tight muons are shown in Fig.~\ref{fig:sfmuon}.

\begin{figure}[ht!]
  \begin{center}
    \includegraphics[width=0.49\textwidth]{fig/efficiency/lepton/muon_eff_tight_id_2017.pdf}
    \includegraphics[width=0.49\textwidth]{fig/efficiency/lepton/muon_eff_loose_id_2017.pdf}\\
    \includegraphics[width=0.49\textwidth]{fig/efficiency/lepton/muon_eff_tight_id_2018.pdf}
    \includegraphics[width=0.49\textwidth]{fig/efficiency/lepton/muon_eff_loose_id_2018.pdf}
    \caption{
      Scale factors for tight (left) and veto (right) muon identification are shown for 2017 (top) and
      2018 (bottom). The scale factors are provided in bins of electron $\pt$ and $\eta$.
    }
    \label{fig:sf_muon_id}
  \end{center}
\end{figure}


\begin{figure}[ht!]
    \begin{center}
        \includegraphics[width=0.49\textwidth]{fig/efficiency/lepton/muon_eff_tight_iso_2017.pdf}
        \includegraphics[width=0.49\textwidth]{fig/efficiency/lepton/muon_eff_loose_iso_2017.pdf}\\
        \includegraphics[width=0.49\textwidth]{fig/efficiency/lepton/muon_eff_tight_iso_2018.pdf}
        \includegraphics[width=0.49\textwidth]{fig/efficiency/lepton/muon_eff_loose_iso_2018.pdf}
        \caption{
            Scale factors for tight (left) and veto (right) muon isolation are shown for 2017 (top) and
            2018 (bottom). The scale factors are provided in bins of electron $\pt$ and $\eta$.
          }
      \label{fig:sf_muon_iso}
    \end{center}
  \end{figure}

\clearpage
\subsection{Higher-order reweighting}\label{sec:nlo}
This analysis uses the ratios of the recoil distributions in signal and control regions to constrain the final background estimate in a partially data driven way. As signal and control regions both have large statistical power, precise predictions of these ratios are necessary. To achieve this goal, the LO simulation samples for the samples W, DY and photon backgrounds are reweighted using higher-order corrections separately corresponding to NLO QCD, NLO EW and NNLO QCD terms. The individual corrections are described in more detail in this section. A concise overview of which corrections are applied to which processes is given in Tab.~\ref{tab:higher_order_summary}.


\begin{table}[ht!]
    \centering
    \small
    \def\arraystretch{1.5}
    \caption{Summary of higher-order corrections applied to simulated samples. For each boson production process, separate samples and corrections are available for the EWK and QCD production modes. ``MC order'' reflects the perturbative order used in the generation of the simulation sample, while the further columns represent corrections applied on a per-event level in the analysis process.}
    \begin{tabular}{c c c c c c}
        \textbf{Boson} & \textbf{production mode}  & \textbf{MC order} & \textbf{NLO QCD} & \textbf{NNLO QCD} & \textbf{NLO EWK} \\\hline\hline
    \multirow{2}{*}{Z} & QCD & LO & \checkmark & \checkmark & \checkmark \\
                       & EWK & LO & \checkmark & -- & -- \\\hline
    \multirow{2}{*}{W} & QCD & LO & \checkmark & \checkmark & \checkmark \\
    & EWK & LO & \checkmark & -- & -- \\\hline
    \multirow{2}{*}{$\gamma$} & QCD & LO & \checkmark & \checkmark & \checkmark \\
    & EWK & LO & -- & -- & -- \\\hline\hline

    \end{tabular}

    \label{tab:higher_order_summary}
\end{table}

\subsubsection{Generator-level boson construction}
All theory-based corrections of the W, DY and photon backgrounds are parametrized as a function of the generator-level \pt of the respective boson \ptv. For each simulated event, this quantity is calculated as follows. For DY and W samples, generator-level dilepton candidates are built from:

\begin{enumerate}
\item ``dressed'' final-state electrons and muons. Lepton dressing means to collect all photons radiated off the lepton within a cone of $\Delta R < 0.1$ and adding their four-momenta back to the lepton four-momentum. This procedure is meant to undo the effect of final state photon radiation, which would otherwise distort the value of the reconstructed boson four-momentum. This effect is especially relevant as electrons and muons follow different radiation patterns. Lepton dressing is performed in central NanoAOD production following the procedure used in the RIVET software.
\item $\tau$ leptons with generator status 2. As $\tau$ leptons are unstable, they are not present as final state particles (status 1) in the generator record. The $\tau$ lepton before its decay has status 2.
\item neutrinos with generator status 1.
\end{enumerate}

The dilepton candidates are checked for flavour consistency with the desired boson candidate. If multiple candidates are found in an event, the one with the highest invariant mass is used.

For photon events, the generator photon with highest \pt and status 1 is used.

\subsubsection{QCD NLO corrections to QCD V processes}

Scale factors corresponding to NLO QCD corrections for W and Z production are obtained from centrally produced CMS samples. For the DY and W processes, samples from the ``Fall17'' campaign are used, while ``Summer16'' samples are used for the $\gamma$+jets process. In both cases, all samples are generated using \texttt{MadGraph5\_amc@NLO}. The LO samples are binned in HT and are equivalent to the ones used in in the analysis, and are generated with up to four partons in the matrix element. The NLO samples are generated with up to two additional partons in the matrix element calculation. Further jet multiplicities are handled by the parton shower, which in both cases is performed using \texttt{Pythia8} with tune \texttt{CP5} (\texttt{CUETP8M1}) for Fall17 (Summer16).

The scale factors are derived by obtaining the distribution of interest at the generator-level in both samples, normalizing the distributions to their respective cross sections, and then dividing them as SF = NLO / LO. Identical selection criteria are applied to both samples based on the generator-level boson and generator-level AK4 jets, which are clustered using all visible generator particles with status 1. Generator-level jets overlapping with a lepton or photon present in the LHE record are rejected. The event selection requirements are:

\begin{enumerate}
\item At least two generator-level jets must be present, with the leading (trailing) \pt of at least 80 GeV (40 GeV) and $|\eta|<4.7$.
\item The angular difference in the transverse plane between the two jets is required to be small: $\Delta\phi(jj) < 1.5$.
\item The two leading jets are required to be well separated in pseudorapidity, $|\eta_{1}-\eta_{2}| > 1$, and be located in opposite hemispheres of the detector, $\eta_{1}\times \eta_{2}<0$.
\item The difference in the azimuthal angle ($\Delta\phi$) between the boson and the four leading jets in the event is required to be larger than 0.5. Only jets with $\pt>30~\GeV$ and $|\eta|<2.4$ are considered.
\end{enumerate}

Compared to an inclusive derivation of the SF, the inclusion of the selection criteria leads to an increase in the value of the SF of approximately five percentage points.


The scale factors are derived with a two-dimensional dependence on \ptv and \mjj, and are shown in Fig.~\ref{fig:theory_sf_qcd_nlo_2d_wz} for W and DY, and in Fig.~\ref{fig:theory_sf_qcd_nlo_2d_gamma} for photon production.

\begin{figure}[ht!]
    \begin{center}
        \includegraphics[width=0.49\textwidth]{fig/theory/2d_dy_gen_vpt_vbf_dress.pdf}
        \includegraphics[width=0.49\textwidth]{fig/theory/2d_wjet_gen_vpt_vbf_dress.pdf} \\
        \caption{
            Same as Fig.~\ref{fig:theory_sf_qcd_nlo}, but now binned in two dimensions of the generator-level boson \pt and \mjj.
            The k factors are derived within the generator-level VBF selection described in the text.
            The uncertainties quoted in each bin are the statistical uncertainties due to the finite size of simulated samples.
          }
      \label{fig:theory_sf_qcd_nlo_2d_wz}
    \end{center}
  \end{figure}

\begin{figure}[ht!]
    \begin{center}
        \includegraphics[width=0.49\textwidth]{fig/theory/2d_gjets_gen_vpt_vbf_stat1.pdf}
        \caption{
            Same as Fig.~\ref{fig:theory_sf_qcd_nlo_2d_wz}, but for photon production.
          }
      \label{fig:theory_sf_qcd_nlo_2d_gamma}
    \end{center}
  \end{figure}

\subsubsection{QCD NNLO corrections to QCD V processes}

NNLO corrections are obtained from the fixed-order calculations results in~\cite{DMTheory}. They are parametrized as a function of the generator-level boson \pt. The correction is shown in Fig.~\ref{fig:theory_sf_qcd_nnlo}.

\begin{figure}[ht!]
    \begin{center}
        \includegraphics[width=0.49\textwidth]{fig/theory/nnlo_qcd.pdf}
        \caption{
            QCD NNLO scale factors for DY, W and photon production as a function of \ptv.
          }
      \label{fig:theory_sf_qcd_nnlo}
    \end{center}
  \end{figure}

\subsubsection{EW NLO corrections to QCD V processes}
Scale factors corresponding to NLO EW corrections are obtained from Ref.~\cite{DMTheory} and applied as a function of the generator-level boson \pt. The scale factors are shown in Fig.~\ref{fig:theory_sf_ew_nlo}.

\begin{figure}[ht!]
    \begin{center}
        \includegraphics[width=0.49\textwidth]{fig/theory/nlo_ewk.pdf}
        \caption{
            EW NLO scale factors for DY, W and photon production as a function of \ptv.
          }
      \label{fig:theory_sf_ew_nlo}
    \end{center}
  \end{figure}

\subsubsection{QCD NLO corrections to EWK V processes}
The QCD NLO corrections to EWK W and Z production have been calculated in Ref.~\cite{AN-2017-267} using the VBF@NLO program. They are parametrized in \ptv and \mjj and are shown in Fig.~\ref{fig:theory_sf_qcd_nlo_for_ewk}.


\begin{figure}[ht!]
    \begin{center}
        \includegraphics[width=0.49\textwidth]{fig/theory/nlo_qcd_for_ewk_dy.pdf}
        \includegraphics[width=0.49\textwidth]{fig/theory/nlo_qcd_for_ewk_w.pdf}
        \caption{
            QCD NLO scale factors for EWK DY, W production of \ptv and \mjj.
          }
      \label{fig:theory_sf_qcd_nlo_for_ewk}
    \end{center}
  \end{figure}