\section{Comparison of V MC: LO+SF vs NLO }
\label{sec:lo_vs_nlo}
In the `Fall17` production campaign, V+jets samples are also available at NLO in QCD. To validate the implementation of the QCD LO $\rightarrow$ NLO rescaling procedure described in sec.~\ref{sec:nlo}, we compare the agreement of data and simulation between the rescaled LO samples (``LO + SF'') and the NLO samples. In both cases, all other background samples are identical. The electroweak NLO and QCD NNLO corrections are applied are applied in both LO + SF and NLO cases.

The resulting recoil shape agreement is shown in Figs.~\ref{fig:lo_vs_nlo_1lep_recoil} and~\ref{fig:lo_vs_nlo_2lep_recoil} for the leptonic control regions. In all regions, LO + SF and NLO samples agree well, indicating that the rescaling procedure works as intended. Residual differences between the two methods are significantly smaller than the overall differences between data and any of the MC-based background models. The comparison is also shown as a function of the leading jet \pt in Figs.~\ref{fig:lo_vs_nlo_1lep_ak4_pt0} and~\ref{fig:lo_vs_nlo_2lep_ak4_pt0}.



\begin{figure}[htbp]
    \begin{center}
        \includegraphics[width=0.49\textwidth]{fig/datamc/cr_1m_j/cr_1m_j_recoil_losf_2017.pdf}
        \includegraphics[width=0.49\textwidth]{fig/datamc/cr_1m_j/cr_1m_j_recoil_nlo_2017.pdf} \\
        \includegraphics[width=0.49\textwidth]{fig/datamc/cr_1e_j/cr_1e_j_recoil_losf_2017.pdf}
        \includegraphics[width=0.49\textwidth]{fig/datamc/cr_1e_j/cr_1e_j_recoil_nlo_2017.pdf} \\
    \end{center}
    \caption{Comparison of LO + SF (left column) and NLO samples (right column) for the single muon (top row) and single electron (bottom row) control regions in the recoil distribution. While the V+jets samples are varied between the left and right columns, all other background samples remain unchanged.}
    \label{fig:lo_vs_nlo_1lep_recoil}
\end{figure}



\begin{figure}[htbp]
    \begin{center}
        \includegraphics[width=0.49\textwidth]{fig/datamc/cr_1m_j/cr_1m_j_ak4_pt0_losf_2017.pdf}
        \includegraphics[width=0.49\textwidth]{fig/datamc/cr_1m_j/cr_1m_j_ak4_pt0_nlo_2017.pdf} \\
        \includegraphics[width=0.49\textwidth]{fig/datamc/cr_1e_j/cr_1e_j_ak4_pt0_losf_2017.pdf}
        \includegraphics[width=0.49\textwidth]{fig/datamc/cr_1e_j/cr_1e_j_ak4_pt0_nlo_2017.pdf} \\
    \end{center}
    \caption{Comparison of LO + SF (left column) and NLO samples (right column) for the single muon (top row) and single electron (bottom row) control regions in the leading jet \pt distribution. While the V+jets samples are varied between the left and right columns, all other background samples remain unchanged.}
    \label{fig:lo_vs_nlo_1lep_ak4_pt0}
\end{figure}


\begin{figure}[htbp]
    \begin{center}
        \includegraphics[width=0.49\textwidth]{fig/datamc/cr_2m_j/cr_2m_j_recoil_losf_2017.pdf}
        \includegraphics[width=0.49\textwidth]{fig/datamc/cr_2m_j/cr_2m_j_recoil_nlo_2017.pdf} \\
        \includegraphics[width=0.49\textwidth]{fig/datamc/cr_2e_j/cr_2e_j_recoil_losf_2017.pdf}
        \includegraphics[width=0.49\textwidth]{fig/datamc/cr_2e_j/cr_2e_j_recoil_nlo_2017.pdf} \\
    \end{center}
    \caption{Same as Fig.~\ref{fig:lo_vs_nlo_1lep_recoil} but for double electron and muon regions. Comparison of LO + SF (left column) and NLO samples (right column) for the double muon (top row) and double electron (bottom row) control regions in the recoil distribution. While the V+jets samples are varied between the left and right columns, all other background samples remain unchanged.}
    \label{fig:lo_vs_nlo_2lep_recoil}
\end{figure}


\begin{figure}[htbp]
    \begin{center}
        \includegraphics[width=0.49\textwidth]{fig/datamc/cr_2m_j/cr_2m_j_ak4_pt0_losf_2017.pdf}
        \includegraphics[width=0.49\textwidth]{fig/datamc/cr_2m_j/cr_2m_j_ak4_pt0_nlo_2017.pdf} \\
        \includegraphics[width=0.49\textwidth]{fig/datamc/cr_2e_j/cr_2e_j_ak4_pt0_losf_2017.pdf}
        \includegraphics[width=0.49\textwidth]{fig/datamc/cr_2e_j/cr_2e_j_ak4_pt0_nlo_2017.pdf} \\
    \end{center}
    \caption{Same as Fig.~\ref{fig:lo_vs_nlo_1lep_ak4_pt0} but for double electron and muon regions. Comparison of LO + SF (left column) and NLO samples (right column) for the double muon (top row) and double electron (bottom row) control regions in the leading jet \pt distribution. While the V+jets samples are varied between the left and right columns, all other background samples remain unchanged.}
    \label{fig:lo_vs_nlo_2lep_ak4_pt0}
\end{figure}
